\documentclass[11pt,letterpaper]{article}

\usepackage{polyglossia}

\setmainlanguage[variant=usmax]{english}
\setotherlanguage{spanish}

\usepackage{fontspec}

\usepackage[backend=biber,backref=true,style=authoryear]{biblatex}

\usepackage{csquotes}

\addbibresource{seth.bib}

\DeclareFieldFormat[article]{title}{#1}
\DeclareFieldFormat[incollection]{title}{#1}
\DeclareFieldFormat[thesis]{title}{#1}

\usepackage{amsmath}
\usepackage{amsthm}

\usepackage{xspace}

\usepackage[xetex]{hyperref}

\hypersetup{hidelinks}

\usepackage{fancyvrb}

\DefineShortVerb{\|}
\DefineVerbatimEnvironment{code}{Verbatim}{frame=lines}

%%%%%%%%%%%%%%%%%%%%%%%%%%%%%%%%%%%%%%%%%%%%%%%%%%%%%%%%%%%%%%%%%%%%%%%%%%%%%%

\theoremstyle{definition}

\newtheorem{definition}{Definition}[section]
\newtheorem{example}{Example}[section]
\newtheorem{note}{Note}[section]
\newtheorem{remark}{Remark}[section]

%%%%%%%%%%%%%%%%%%%%%%%%%%%%%%%%%%%%%%%%%%%%%%%%%%%%%%%%%%%%%%%%%%%%%%%%%%%%%%

\DeclareMathOperator{\obj}{O}
\DeclareMathOperator{\mor}{M}

\DeclareMathOperator{\dom}{dom}
\DeclareMathOperator{\cod}{cod}

\DeclareMathOperator{\id}{id}

\newcommand{\idO}[1]{\natO{\id}{#1}}

\newcommand{\comp}{\ensuremath{\mathrel{\circ}}}

\newcommand{\cat}[1]{\ensuremath{\mathcal{#1}}}

\newcommand{\catbf}[1]{\ensuremath{\mathbf{#1}}\xspace}

\newcommand{\hask}{\catbf{Hask}}
\newcommand{\set}{\catbf{Set}}

\newcommand{\func}[1]{\ensuremath{\mathsf{#1}}}

\newcommand{\funcO}[1]{\ensuremath{\func{#1}_{\obj}}}
\newcommand{\funcM}[1]{\ensuremath{\func{#1}_{\mor}}}

\newcommand{\nat}[1]{\ensuremath{#1}}

\newcommand{\natO}[2]{\ensuremath{\nat{#1}_{#2}}}

%%%%%%%%%%%%%%%%%%%%%%%%%%%%%%%%%%%%%%%%%%%%%%%%%%%%%%%%%%%%%%%%%%%%%%%%%%%%%%

\begin{document}

\title{Category Theory Applied to Functional Programming}

\author{Juan Pedro Villa Isaza\thanks{
    \href{mailto:jvillai@eafit.edu.co}{\nolinkurl{jvillai@eafit.edu.co}}.}}

\date{2014}

\maketitle

\begin{abstract}

  We study some of the applications of category theory to functional
  programming in the context of the Haskell functional programming
  language. In particular, we describe and explain the concepts of
  category theory which allow us to conceptualize and better
  understand Haskell functors: categories, functors, and endofunctors.

  \vspace{1em}
  \noindent
  Keywords: category, category theory, endofunctor, functional
  programming, functor, Haskell.

\end{abstract}

\begin{spanish}

  \begin{center}

    {\Large Teoría de categorías aplicada a la programación funcional}

  \end{center}

  \begin{abstract}

    Estudiamos algunas de las aplicaciones de la teoría de categorías
    a la programación funcional en el marco del lenguaje de
    programación funcional Haskell. En particular, describimos y
    explicamos los conceptos de teoría de categorías que permiten
    conceptualizar y profundizar en los funtores de Haskell:
    categorías, funtores y endofuntores.

    \vspace{1em}
    \noindent
    Palabras clave: categoría, endofuntor, funtor, Haskell,
    programación funcional, teoría de categorías.

  \end{abstract}

\end{spanish}

%%%%%%%%%%%%%%%%%%%%%%%%%%%%%%%%%%%%%%%%%%%%%%%%%%%%%%%%%%%%%%%%%%%%%%%%%%%%%%

\section{Introduction}
\label{sec:introduction}

Category theory is a branch of mathematics developed in the 1940s
which has come to occupy a central position in computer science, and,
in particular, in functional programming. Broadly, it is a convenient
conceptual framework based on the concepts of categories, functors,
and natural transformations
\parencites[vii]{maclane-1998}[1]{marquis-2013}.

According to \textcite[73]{elkins-2009} and
\textcite[50--51]{yorgey-2009}, several concepts of functional
programming languages, such as Agda \parencites{norell-2007}{agda},
Haskell \parencite{peytonjones-2003}, Miranda \parencite{turner-1985},
ML \parencite{milner-1984}, among others, are based on concepts of
category theory, but one can be a perfectly competent functional
programmer without knowledge of these theoretical foundations. In
spite of that, category theory can be applied to functional
programming with the purpose of, for instance, better understanding
concepts such as algebraic data types, functors, monads, and
parametrically polymorphic functions, and thus becoming a better
programmer.

In this regard, we study some of the applications of category theory
to functional programming in Haskell. More specifically, we describe
and explain the concepts of category theory needed for conceptualizing
and better understanding functors in Haskell. With this purpose, we
examine categories, functors, and endofunctors. In other words, we
approach Haskell functors through category theory.

\begin{note}
  \label{note:cain}

  This article is a summary of Category Theory Applied to Functional
  Programming, an undergraduate project supervised by Andrés Sicard
  Ramírez. It is submitted in partial fulfillment of the requirements
  for the degree of Systems Engineering at EAFIT University. For more
  examples and the missing proofs, see \parencite[§§~2,
    4]{villaisaza-2014}.

\end{note}

\begin{note}
  \label{note:haskell}

  The Haskell code was tested with GHC 7.6.3 and the following
  language options:
  \begin{code}
{-# LANGUAGE InstanceSigs   #-}
{-# LANGUAGE KindSignatures #-}
  \end{code}

\end{note}

%%%%%%%%%%%%%%%%%%%%%%%%%%%%%%%%%%%%%%%%%%%%%%%%%%%%%%%%%%%%%%%%%%%%%%%%%%%%%%

\section{Motivation}
\label{sec:motivation}

In Haskell, mapping over lists, which is accomplished with the |map|
function, is a dominant idiom \parencite[146]{lipovača-2011}:
\begin{code}
map :: (a -> b) -> [a] -> [b]
map _ []     = []
map f (x:xs) = f x : map f xs
\end{code}
Clearly, |map|~|f|~|xs| is the list obtained by applying a function
|f| to each element of a list |xs|, which is a weak specification.
Indeed, let us consider the |map'| function
\parencite[22]{yorgey-2009}:
\begin{code}
map' :: (a -> b) -> [a] -> [b]
map' _ []     = []
map' f (x:xs) = f x : f x : map' f xs
\end{code}
Again, |map'|~|f|~|xs| is the list obtained by applying a function |f|
to each element of a list |xs| (and duplicating each result).
Obviously, we should accept |map| and reject |map'|, but for what
reason? The answer lies in the |Functor| type class, which has its
origins in category theory.

%%%%%%%%%%%%%%%%%%%%%%%%%%%%%%%%%%%%%%%%%%%%%%%%%%%%%%%%%%%%%%%%%%%%%%%%%%%%%%

\section{Categories}
\label{sec:categories}

We define the concept of categories in order to be able to define the
concepts of functors and endofunctors.

\begin{definition}
  \label{def:category}

  %% See \parencite[Definition 2.1]{villaisaza-2014}.

  A category \cat{C} consists of:
  \begin{itemize}
  \item
    Objects $a$, $b$, $c$, ...
  \item
    Morphisms or arrows $f$, $g$, $h$, ...
  \item
    For each morphism $f$, domain and codomain objects $a = \dom(f)$
    and $b = \cod(f)$, respectively. We then write $f: a \to b$.
  \item
    For each object $a$, an identity morphism $\idO{a}: a \to a$.
  \item
    For each pair of morphisms $f: a \to b$ and $g: b \to c$, a
    composite morphism $g \comp f: a \to c$.
  \end{itemize}
  The category is subject to:
  \begin{itemize}
  \item
    Associativity: for all morphisms $f: a \to b$, $g: b \to c$, and
    $h: c \to d$,
    \begin{equation*}
      h \comp (g \comp f) = (h \comp g) \comp f
      \text{.}
    \end{equation*}
  \item
    Identity: for all morphisms $f: a \to b$,
    \begin{equation*}
      \idO{b} \comp f = f = f \comp \idO{a}
      \text{.}
    \end{equation*}
  \end{itemize}

\end{definition}

As examples, we consider \set, the category of sets and functions, and
\hask, a category of Haskell types and functions.

\begin{example}
  \label{ex:set}

  %% See \parencite[Example 2.1.3]{villaisaza-2014}.

  \set is the category with sets $A$, $B$, $C$, ... as objects and
  functions $f$, $g$, $h$, ... as morphisms. Each function $f: A \to
  B$ is composed of a domain $A = \dom(f)$, a codomain $B = \cod(f)$,
  and a rule which assigns to each element $x \in A$ an element $f(x)
  \in B$. For each set $A$, there is an identity function $\idO{A}: A
  \to A$ such that, for all $x \in A$, $\idO{A}(x) = x$, and, for each
  pair of morphisms $f: A \to B$ and $g: B \to C$, there is a
  composite function $g \comp f: A \to C$ such that, for all $x \in
  A$, $(g \comp f)(x) = g(f(x))$.

\end{example}

\begin{remark}
  \label{re:foundations}

  %% See \parencite[Remark 2.2]{villaisaza-2014}.

  To some extent, we are considering the objects of \set to be the set
  of all sets, which would lead us to a paradox such as the set of all
  sets not members of themselves. For this reason, we ought to assume,
  for instance, that there is a big enough set, the universe, and take
  the objects of \set to be the sets which are members of the
  universe. For a full account on mathematical foundations of category
  theory, see \parencites[§~1.8]{awodey-2010}[§~I.6]{maclane-1998}.

\end{remark}

\begin{example}
  \label{ex:hask}

  %% See \parencite[§ 2.2]{villaisaza-2014}.

  \hask is the category with Haskell types and functions as objects
  and morphisms, respectively. By Haskell types, we mean type
  expressions with kind |*|, like |Bool|:
  \begin{code}
data Bool = False | True
  \end{code}
  and, by Haskell functions, functions between such types, like |not|:
  \begin{code}
not :: Bool -> Bool
not False = True
not True  = False
  \end{code}
  Identity functions are given by the |id| function:
  \begin{code}
id :: a -> a
id x = x
\end{code}
  and composite functions, by the |(.)| function:
\begin{code}
(.) :: (b -> c) -> (a -> b) -> a -> c
(g . f) x = g (f x)
  \end{code}

\end{example}

\begin{remark}
  \label{re:hask}

  %% See \parencite[Convention 1]{villaisaza-2014}.

  Unfortunately, Haskell types have bottom. For instance, the values
  of |Bool| include not only |False| and |True|, but also |undefined|:
  \begin{code}
undefined :: a
undefined = undefined
  \end{code}
  As a consequence, Haskell types do not yield a category. Thus, by
  Haskell types, we actually mean Haskell types without bottom, which
  yield a category: \hask.

\end{remark}

\begin{note}
  \label{note:categories}

  The definition of category, the example of \set, and the remark
  about foundations are based on \parencites[4--5]{awodey-2010}[7--9,
    12, 21, 289]{maclane-1998}. For more information about \hask, see,
  for instance, \parencites[74]{elkins-2009}[22, 49--51]{yorgey-2009}.

\end{note}

%%%%%%%%%%%%%%%%%%%%%%%%%%%%%%%%%%%%%%%%%%%%%%%%%%%%%%%%%%%%%%%%%%%%%%%%%%%%%%

\section{Functors}
\label{sec:functors}

Having defined the concept of categories, let us now define the
concept of functors or morphisms of categories. Basically, a functor
is an assignment of the objects and morphisms of a category to objects
and morphisms of another category, such that the identity and
composite morphisms are preserved.

\begin{definition}
  \label{def:functor}

  %% \parencite[Definition 4.1]{villaisaza-2014}

  Let \cat{C} and \cat{D} be categories. A functor $\func{F}: \cat{C}
  \to \cat{D}$ assigns to each object $a$ in \cat{C} an object
  $\funcO{F}(a)$ in \cat{D}, and to each morphism $f: a \to b$ in
  \cat{C} a morphism $\funcM{F}(f): \funcO{F}(a) \to \funcO{F}(b)$ in
  \cat{D}, such that, for all objects $a$ in \cat{C},
  \begin{equation}
    \label{eq:functor-identity}
    \funcM{F}(\idO{a}) = \idO{\funcO{F}(a)}
    \text{,}
  \end{equation}
  and, for all morphisms $f: a \to b$ and $g: b \to c$ in \cat{C},
  \begin{equation}
    \label{eq:functor-composition}
    \funcM{F}(g \comp f) = \funcM{F}(g) \comp \funcM{F}(f)
    \text{.}
  \end{equation}
  Moreover, an endofunctor is a functor from a category to itself.

\end{definition}

A simple example is the power set endofunctor.

\begin{example}
  \label{ex:functor-power-set}

  %% See \parencite[Example 4.1.1]{villaisaza-2014}.

  The power set endofunctor $\func{P}: \set \to \set$ assigns to each
  set $A$ the set of all subsets of $A$, that is, $\funcO{P}(A) = \{X
  \mid X \subseteq A\}$, and to each function $f: A \to B$ a function
  $\funcM{P}(f): \funcO{P}(A) \to \funcO{P}(B)$ such that, for all $X
  \in \funcO{P}(A)$, $\funcM{P}(f)(X) = \{f(x) \mid x \in X\}$.

\end{example}

\subsection{Functors in Haskell}
\label{sec:functors-haskell}

In Haskell, functors are defined by the |Functor| type class:
\begin{code}
class Functor f where
  fmap :: (a -> b) -> f a -> f b
\end{code}
This type class consists of a type constructor |f|, that is, a type
expression with kind |*|~|->|~|*|, like |Maybe| (see Example
\ref{ex:functor-maybe}), and an |fmap| function, which is curried.
More precisely, specifying the kind signature of |f| and adding
unnecessary parentheses to the type signature of |fmap|:
\begin{code}
class Functor (f :: * -> *) where
  fmap :: (a -> b) -> (f a -> f b)
\end{code}

Therefore, a Haskell functor consists of a type constructor |f|, which
assigns to each type |a| a type |f|~|a|, and an |fmap| function, which
assigns to each function |a|~|->|~|b| a function |f|~|a|~|->|~|f|~|b|,
such that \eqref{eq:functor-identity} and
\eqref{eq:functor-composition} hold:
\begin{code}
fmap id      = id
fmap (g . f) = fmap g . fmap f
\end{code}

Hence, Haskell functors are endofunctors in \hask. As examples, we
consider the |Maybe| and |[]| (list) functors.

\begin{example}
  \label{ex:functor-maybe}

  %% \parencite[Example 4.2.2]{villaisaza-2014}

  In \hask, the |Maybe| functor consists of the |Maybe| type
  constructor:
  \begin{code}
data Maybe a = Nothing | Just a
  \end{code}
  which assigns to each type |a| a type |Maybe|~|a|, and an |fmap|
  function defined as follows:
  \begin{code}
instance Functor Maybe where
  fmap :: (a -> b) -> Maybe a -> Maybe b
  fmap _ Nothing  = Nothing
  fmap f (Just x) = Just (f x)
  \end{code}

\end{example}

\begin{example}
  \label{ex:functor-list}

  %% \parencite[Example 4.2.3]{villaisaza-2014}

  In \hask, the |[]| functor consists of the |[]| type constructor:
  \begin{code}
data [] a = [] | a : [a]
  \end{code}
  which assigns to each type |a| a type |[]|~|a| or |[a]|, and the
  |map| function (see Section \ref{sec:motivation}):
  \begin{code}
instance Functor [] where
  fmap :: (a -> b) -> [a] -> [b]
  fmap = map
  \end{code}

  However, since the |Functor| type class does not include
  \eqref{eq:functor-identity} and \eqref{eq:functor-composition}, the
  |[]| type constructor and the |map'| function (see Section
  \ref{sec:motivation}) constitute another |[]| (Haskell) functor:
  \begin{code}
instance Functor [] where
  fmap :: (a -> b) -> [a] -> [b]
  fmap = map'
  \end{code}
  but this is not an endofunctor in \hask because identity functions
  are not preserved. For instance:
  \begin{code}
> map' id [False]
[False,False]
> id [False]
[False]
  \end{code}

  So, to sum up, we should accept |map| and reject |map'| because only
  the former preserves identity and composite functions, or, in other
  words, the latter is not a uniform action over lists.

\end{example}

\begin{note}
  \label{note:functors}

  The definition of functor and the example of the power set
  endofunctor are based on
  \parencites[13]{maclane-1998}[10--11]{marquis-2013}[428,
    431]{poigne-1992}, and the study of functors in Haskell, on
  \parencites[146--150, 218--227]{lipovača-2011}[18--23]{yorgey-2009}.

\end{note}

%%%%%%%%%%%%%%%%%%%%%%%%%%%%%%%%%%%%%%%%%%%%%%%%%%%%%%%%%%%%%%%%%%%%%%%%%%%%%%

\section{Conclusions}
\label{sec:conclusions}

We described and explained categories, functors, and endofunctors, and
applied them to Haskell and, in particular, Haskell functors and the
|map| function, which we used as motivation. While studying category
theory may not be necessary for mapping over lists, we believe it is
very useful for better understanding what that means. In addition,
even though it might be difficult at first (as a matter of fact,
\textcite[25]{bird-demoor-1997} claim that ``one does not so much
learn category theory as absorb it over a period of time''), we claim
it is definitely worth it.

Obviously, we did not cover all of category theory. For instance, we
did not include concepts such as algebras and initial algebras, monads
and Kleisli triples, and natural transformations. Still, the notions
of categories and functors offer an appropriate starting point for the
study of category theory applied to functional programming, and give a
general idea of \parencite{villaisaza-2014}, in which we consider
functors, parametrically polymorphic functions, monads, and algebraic
data types through category theory.

%%%%%%%%%%%%%%%%%%%%%%%%%%%%%%%%%%%%%%%%%%%%%%%%%%%%%%%%%%%%%%%%%%%%%%%%%%%%%%

\printbibliography

%%%%%%%%%%%%%%%%%%%%%%%%%%%%%%%%%%%%%%%%%%%%%%%%%%%%%%%%%%%%%%%%%%%%%%%%%%%%%%

\end{document}
